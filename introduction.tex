\section{Introduction}
\label{sec:intro}

Angry Birds is a fun 2D video game, released in 2009, which was well received by the general public. Its colorful and fun interface and free basic use of
the program. The plot of the game is simple: It is about a group of birds that are angry with pigs which have stolen the bird's eggs. The game asks the player to throw birds at pigs with the goal of hitting them. The game consists of many levels, and for each level there is a fixed number of birds that can be launched by means of a slingshot in a predefined order. The goal of the game is hitting every pig present in each level. Different types of birds have special abilities. The game gradually gets harder as pigs cover under shelters made of different types of materials like wood, stone and glass. A level is completed if every pig is hit and the score of each level is obtained by considering the number of obstacles destroyed and the number of birds not used.
The \abc~\cite{angryAI} with the ultimate goal of developing an artificial agent that outperforms any human was created to stimulate a scientific treatment of the problem. While the challenge might sound easy at first it has revealed many interesting sub-problems and a multitude of possible approaches which lead to numerous scientific publications.
Sub-problems include, but are not limited to, finding tactics that yield high scores and developing computer vision that can correctly perceive the game environment in a language suitable for the agent.

Our work is based on the agent called \ah \cite{angryhex}. Over the last five years of the competition it achieved great results and was continually refined \footnote{For competition results consider \url{https://aibirds.org/past-competitions.html}}: 2013 (1\textsuperscript{st}), 2014 (7\textsuperscript{th}), 2015 (2\textsuperscript{nd}), 2016 (2\textsuperscript{nd}), 2017 (3\textsuperscript{rd}). Its reasoning is based on a logic programming approach, which models game knowledge and the game environment by means of \emph{Answer Set Programming} (ASP).

% TODO: Do the \ah people actually explain why they chose HEX-programs? Referring this reason here would be good.
However, the authors chose not to rely on common ASP in first-order logic, but implemented the reasoning core in an ASP variant, so called HEX-programs~\cite{hex}. Thus, some rules are formulated in higher-order logic (quantifying over predicates).

The initial motivation behind this work was to modify \ah such that it may use the \al ASP system~\cite{blend} for reasoning. However, \al --- and not many ASP systems in general --- does not admit higher order atoms. Therefore, the HEX-program(s) that power \ah were rewritten to first-order logic. In order to achieve that, higher-order atoms were exchanged for a more complex domain, which not only contains symbolic constants, integers, and strings but any Java object. This approach has some advantages as programs (both ASP programs and implementations of external atoms) tend to be more readable.

This work is structured as follows: First we have give preliminary notions, mostly on logic programming in general and HEX-programs in particular, in Section \ref{sec:prelim}. Then we introduce the architecture of the \ah agent in Section \ref{sec:agent} which makes the connection to the \abc. The translation process we applied to obtain ASP programs with external atoms from more general HEX-programs is subject of Section \ref{sec:main}, and we  conclude in the last section.