\section{The \ah Agent}
\label{sec:agent}
% Describe the Angry Birds scenario/setting and how the agent works
\ah is an artificial player for \ab,
which reasoning with respect to
the world knowledge is carried out by
means of computing the answer set of a
HEX-program.
It originated as a re-implementation of
the na\"{i}ve agent provided by the organizers
of the \abc \cite{angryAI},
with general improvements on main components
of it and a complete rewriting of the
\emph{planning} component as a collection
of HEX-programs.

In this section, an overview of the \ah
agent will be presented, with a focus on
the declarative part and its features. 
For a thorough and detailed explanation of
the implementing process, of the agent
layout and additional considerations on
performance, the reader may refer to~\cite{angryhex}.

\subsection{Architecture of the Agent and its Connection to the Game}
\label{sec:agent_base}

The system that executes \ah and \ab consists of a
number of loosely coupled components,
each dealing with different aspects of the interaction.
Since the game itself runs inside a browser a
\emph{browser extension} that takes images of the game
and forwards input to it is connected to a \emph{game server}
which mediates between the agent and the game. The game server
is in turn connected to \ah.
The competition organizers prepared some Java code to
simplify bootstrapping an agent implementation, and \ah
makes use of these components, also referred to as \emph{framework}.

Aside from all this auxiliary machinery we are interested in
the core components of the agent that are closely connected to
or implement its reasoning process:
\begin{description}
    \item[Vision] segments the images
    captured from the game environment, returning
    the minimum bounding rectangles of essential
    objects, as well as relevant information,
    like the types of birds available, the material
    of which bricks are composed and where pigs are placed.
    It is part of the framework.
    \item[Trajectory] estimates
    the flight path of birds as they are
    catapulted off the slingshot; here, orientation and distance of
    the release point are taken into account.
    \item[Planning] also called \emph{AI agent} in the
    % What the hell is "the original setting"? Maybe write it in quotes since its just a reference but we are not actually using that term.
    original setting, this part delivers the order
    of the played levels, the choice of birds to use,
    and other strategy-related choices.
\end{description}

\subsection{HEX-programs for Planning the Agent's Actions}

The main contribution of \ah is the \textbf{Planning} component, which has
been rewritten as a collection of HEX-programs.
These are partitioned in two layers, namely
the \textbf{Tactic} and the \textbf{Strategy} layer.
Then, these programs are fed to the
\textsc{dlv-hex} solver~\cite{dlvHEX},
which returns answer sets containing
information about the next moves in the game.

The purpose of the \textbf{Tactic} layer is to compute optimal shots, retrieving information from the current scene in the game and using knowledge modeled by means of a HEX-program \(\mathcal{P}_{AI}\).
This program represents the knowledge of the agent on shootable targets, the estimates of the damage occurring when a target is hit and the priority of the targets, according to the estimated damages.

The \emph{input} of the \textbf{Tactic} layer consists of the program \(\mathcal{P}_{AI}\), together with a collection \(\mathcal{S}\) of logic assertions about the scene, provided by the \textbf{Vision} component.
The \emph{output} of the layer corresponds to the answer sets of \(\mathcal{P}_{AI} \cup \mathcal{S}\), which contain information about which target to hit and what shot is required.

\emph{External atoms} are used in \(\mathcal{P}_{AI}\) mainly to outsource physics-based simulations to a dedicated, external program and to obtain insights about possible shots, consequences of the shots and the geometry of the scene.

The task of the \textbf{Strategy} layer, on the other hand, is to schedule the sequence of played levels. This layer proceeds according to the following scheme:
\begin{enumerate}
    \item First, all available levels are played once;
    \item Then, each level in which the obtained scores has the greatest difference from the current best score is selected --- up to a set threshold of possible attempts;
    \item Next, those levels where \ah scored better than the current best score, with the least difference in score, are selected (up to a certain number of attempts);
    \item To break ties, a random level is selected.
\end{enumerate}

The \textbf{Strategy} layer keeps a track of the scores that were previously achieved and of the objects that were previously chosen as initial targets.
Objects that were previously targeted are excluded; this ensures that the agent is going to employ a new strategy, with each new attempt on the same level.