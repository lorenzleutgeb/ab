\section{Complex Predicates~vs.~Complex Objects}
\label{sec:main}

In this section we present how a specific class of occurences of higher order predicates can be translated to first order by introducing more complex terms.

For example, consider the following rule taken from the tactics of \ah:
% https://github.com/DeMaCS-UNICAL/Angry-HEX/blob/4ea72273518dfde6240ece9dadea927fd1b79c3c/dlv/tactic.dlv#L133
$$ canPush(o_a,o_b) \leftarrow \&canPush[objects,hills](o_a,o_b). \label{main:rule-1} $$

Intuitively, $canPush(o_a, o_b)$ should be true if the object $o_a$ can push the object $o_b$.
To establish the truth of $canPush(o_a, o_b)$ spatial information, like minimum distance and geometric range of the objects, is required; this is obtained by outsourcing the computation to an external source, by means of the external predicate $\&canPush$.
The required knowledge is given as input to $\&canPush$ by using the predicates $objects$ and $hills$: indeed, $\&canPush$ is a \emph{higher-order predicate}.

A useful observation is that the predicates $objects$ and $hills$ both represent sets: $objects$ characterizes the collection of objects that are present in the current level, while $hills$ encodes the morphological structure of the ground on the current level.
These collections represent a form of knowledge that the agent gains before the reasoning phase.
Indeed, in \ah the objects and the information about the world scene are first acquired by the \textbf{Vision} component, as explained in Section~\ref{sec:agent_base}; only after this phase, the \textbf{Planning} component is invoked and the reasoning process takes place.
Thus, for the evaluation of the answer sets w.r.t.\ $canPush$, the elements for which $objects$ and $hills$ do not vary: they are encoded as facts, in the set of assertions \(\mathcal{S}\) added to the program \(\mathcal{P}_{AI}\).

Once understood that $objects$ and $hills$ are representations of sets, we may pose the following question: Would it be possible to replace these predicates with sets?
Or phrased differently: s there a way to have two sets $O$ and $H$, respectively the sets of objects and hills, in place of  $objects$ and $hills$ in the call to $\&canPush$?

The ASP system underlying \ah does not provide support for complex data structures as objects --- thus, sets as objects are not allowed ---, but only symbolic constants, strings and natural numbers.
By lifting such a restriction, we may be able to translate the rule~\eqref{main:rule-1}, containing a higher-order predicate, into one where only first-order predicates appear, such as:
$$ canPush(o_a,o_b) \leftarrow objects(O), \ hills(H), \ \&canPush[O,H](o_a,o_b). \label{main:rule-2} $$

Now, the predicates $objects$ and $hills$ are unary predicates with a different role: their interpretation will only contain one complex object, a set.
Applying this kind of translation to all the rules employing higher-order predicates yields a program where only first-order predicates appear.

\subsection{Translation}

\ah has been initially implemented to work with the \textsc{dlv-hex} solver~\cite{dlvHEX}, which successfully handles higher-order predicates.
Our ASP system of choice, \al (introduced in~\cite{blend}), despite lacking this feature, extends the range of supported data structures beyond those included in \textsc{dlv-hex} --- symbolic constants, natural numbers and strings --- by allowing for terms to be interpreted as Java objects.
The proposed translation exploits this feature: if one can convert the information conveyed by the inputs of the external, higher-order predicates into Java objects, then the higher-order predicates are not necessary anymore.

The first phase of the translation consists of the replacements of input predicates into Java objects, as mentioned in the previous paragraph.
In our ongoing example, the two predicates $objects$ and $hills$, representing sets, are replaced by a java class called \texttt{ASPWorld}.
Each instance of this class represents a snapshot of the world, as retrieved by the agent from the current level.
The class contains two \texttt{Collection} variables: one is expected to contain Angry Birds objects (of type \texttt{ABObject}), while the other includes polygonal shapes, in place of $hills$.

Once that the interpretation of the input of the external predicate has been encoded as a Java class, the interpretation of the external predicate has to be defined.
In \al, this is achieved by means of a Java method.
For $\&canPush$, the interpretation is  is specified as a \texttt{@Predicate} instance with type signature
\begin{center}
    \texttt{public Set<List<ConstantTerm>\-> canPush(ASPWorld aspWorld)}
\end{center}
which indicates that the predicate returns a list of constant terms (of the form \((A,B)\)), that represent the fact that \(A\) can push \(B\), after being hit.

Finally, it is possible to rewrite the ASP program, eliminating all the sources of higher-order predicates.
In our setting, the clause containing the higher-order predicate in~\eqref{main:rule-1} is replaced by the clause

$$ canPush(A,B) \leftarrow world(W), \&canPush[W](A,B) \label{main:rule-3} $$

Here, $\&canPush$ is no longer a higher-order predicate: indeed, it receives a ground atom $W$ as input --- which is interpreted as a \texttt{ASPWorld} instance --- and returns a list of tuples of constant terms $(A,B)$ in which the item $A$ can effectively push the item $B$.

\subsection{About the translation}

While one may be tempted to label the described procedure as a general technique to achieve second-order elimination, it may be useful to investigate what made the proposed conversion possible and to discuss whether it is always the case for such a translation to be applicable.

In the setting of \ah, the use of higher-order predicates was motivated by the need to collect multiple items of the same kind (birds, bricks, pigs, \ldots) as input of the external predicates.
Since input predicates were used as collectors --- hence, used as auxiliary data structures --- the availability of terms' interpretation as Java classes, provided by \al, made it possible to resort to first-order predicates with terms interpreted as suitable Java classes.

However, this approach may not be applicable to other settings, where higher-order predicates are used in a different way.
Thus, we do not claim this to be a general technique for second-order elimination.

\subsection{Complexity}

While higher-order atoms do not increase the complexity of the reasoning process --- compared to ordinary atoms--- as described in~\cite{hex}, external atoms may be a source of additional complexity.
Given that the HEX-program $\mathcal{P}_{AI}$ used in the \textbf{Planning} component is disjunction-free, the result stated in~\cite[Theorem 8]{hex} let us conclude that deciding whether $\mathcal{P}_{AI}$ has some answer sets is in $NEXP^C$, where $C$ is the complexity class of the functions associated with the external atoms.
In our case, all the Java methods used as interpretation for the external predicates can be seen as functions with complexity in $NP$: this lets us conclude that the complexity of the reasoning process carried out by the \textbf{Planning} component is $NEXP^{NP}$.
