\documentclass{llncs}

% TODO:
% 10 oages
% Because there are no publications about external atoms in Alpha, we may reference DLV-Hex and say that external
% atoms are supported in a similar manner.
%
% A little example showcasing external atoms should be provided.
%
% Formalize "down lifting" of higher order predicates and describe it.
% Its probably is fair to introduce ASP.

\usepackage[utf8]{inputenc}
\usepackage[T1]{fontenc}

\usepackage{hyperref}
\usepackage{graphicx}

% Needs splncs03.bst via Springer
\bibliographystyle{splncs03}

\usepackage{amsmath, amssymb} % For \flalign, \top, ...
\usepackage[ruled,linesnumbered]{algorithm2e}

\usepackage[table,usenames,dvipsnames]{xcolor}
\usepackage{nicefrac,xspace,csquotes}

\newcommand{\fail}{\mathrm{not } \ \xspace}
%\newcommand{\from}{\mathrm{\ \xspace :- \ \xspace}}
\newcommand{\from}{\ensuremath{\leftarrow}}

\newcommand{\entails}{\models}

% Least model (of a Horn program).
\newcommand{\lm}{\mathrm{lm}}

% Set of stable models (of a program).
\newcommand{\stm}{\mathrm{STM}}
\newcommand{\sol}{\mathrm{Sol}}
\newcommand{\compl}{\mathrm{Co}}
\newcommand{\groundext}{\mathrm{Gr}}
\newcommand{\defense}{\mathrm{Def}_F}

% Entails according to well founded semantics.
\newcommand{\wf}{\ensuremath{\entails_{wf}}}

% Entails according to stable model semantics using brave reasoning
\newcommand{\brave}{\ensuremath{\entails_{st}^b}}

% Entails according to stable model semantics using cautious reasoning
\newcommand{\caut}{\ensuremath{\entails_{st}^c}}

% Selective Linear Definite-clause with Negation as Failure
\newcommand{\sldnf}{\ensuremath{\vdash_{NF}}}

\newcommand{\universe}{\mathcal{U}}
\newcommand{\afs}{\mathcal{F}}
\newcommand{\attacks}{\rightsquigarrow}

\newcommand{\ah}{Angry-HEX\xspace}

\title{Complex Predicates~vs.~Complex Objects}
\subtitle{A Case Study on Answer Set Programming\\ for Implementing Artificial Agents}
\author{Filippo~De~Bortoli \and Lorenz~Leutgeb \and Cosimo~Damiano~Persia}
\institute{International Center for Computational Logic, TU Dresden\\ \email{\{filippo.de\_bortoli,lorenz.leutgeb,cosimo\_damiano.persia\}@mailbox.tu-dresden.de}}

\begin{document}

\maketitle

\begin{abstract}
Angry Birds is a 2D video game in which objects are hurled at targets using a slingshot in order to score points. The game sparked interest in the research community and a variety of agents have been developed.
In the fashion of a case study, this work considers on of these agents, namely \ah, which reasons about playing strategy and tactics using \emph{Answer Set Programming} (ASP), a well established formalism for \emph{Knowledge Representation and Reasoning} (KRR).
In order to interface with the game (a web application), and to integrate simulation of physics into the reasoning process, the implementation of \ah makes use of both \emph{higher-order} atoms as well as \emph{external atoms}.
We argue that one of the motivations to use higher-order atoms is imposed by the restrictions of the underlying ASP system. In particular, only domain objects of primitive data types can be used to model complex objects that are relevant in the context of the game.
Consequently we adapt \ah for a setting where higher order predicates have been traded for more complex objects. The resulting agent is dubbed Angry-Alpha.
\end{abstract}

\begin{center}
\begin{tabular}{p{0.2\textwidth}p{0.8\textwidth}}
\bfseries{Keywords:} & Answer Set Programming (ASP), HEX-programs, Knowledge Representation and Reasoning (KRR), Modelling\\
\end{tabular}
\end{center}

\section{Introduction}
\label{sec:intro}

Angry Birds is a fun 2D video game, released in 2009, which was well received by the general public. Its colorful and fun interface and free basic use of
the program. The plot of the game is simple: It is about a group of birds that are angry with pigs which have stolen the bird's eggs. The game asks the player to throw birds at pigs with the goal of hitting them. The game consists of many levels, and for each level there is a fixed number of birds that can be launched by means of a slingshot in a predefined order. The goal of the game is hitting every pig present in each level. Different types of birds have special abilities. The game gradually gets harder as pigs cover under shelters made of different types of materials like wood, stone and glass. A level is completed if every pig is hit and the score of each level is obtained by considering the number of obstacles destroyed and the number of birds not used.
The \abc~\cite{angryAI} with the ultimate goal of developing an artificial agent that outperforms any human was created to stimulate a scientific treatment of the problem. While the challenge might sound easy at first it has revealed many interesting sub-problems and a multitude of possible approaches which lead to numerous scientific publications.
Sub-problems include, but are not limited to, finding tactics that yield high scores and developing computer vision that can correctly perceive the game environment in a language suitable for the agent.

Our work is based on the agent called \ah \cite{angryhex}. Over the last five years of the competition it achieved great results and was continually refined \footnote{For competition results consider \url{https://aibirds.org/past-competitions.html}}: 2013 (1\textsuperscript{st}), 2014 (7\textsuperscript{th}), 2015 (2\textsuperscript{nd}), 2016 (2\textsuperscript{nd}), 2017 (3\textsuperscript{rd}). Its reasoning is based on a logic programming approach, which models game knowledge and the game environment by means of \emph{Answer Set Programming} (ASP).

% TODO: Do the \ah people actually explain why they chose HEX-programs? Referring this reason here would be good.
However, the authors chose not to rely on common ASP in first-order logic, but implemented the reasoning core in an ASP variant, so called HEX-programs~\cite{hex}. Thus, some rules are formulated in higher-order logic (quantifying over predicates).

The initial motivation behind this work was to modify \ah such that it may use the \al ASP system for reasoning \cite{blend}. However, \al --- and not many ASP systems in general --- does not admit higher order atoms. Therefore, the HEX-program(s) that power \ah were rewritten to first-order logic. In order to achieve that, higher-order atoms were exchanged for a more complex domain, which not only contains symbolic constants, integers, and strings but any Java object. This approach has some advantages as programs (both ASP programs and implementations of external atoms) tend to be more readable.

This work is structured as follows: First we will give preliminary notions, mostly on logic programming in general and HEX-programs in particular, in Section \ref{sec:prelim}. Then we introduce the architecture of the \ah agent in Section \ref{sec:agent} which makes the connection to the \abc. The translation process we applied to obtain ASP programs with external atoms from more general HEX-programs is subject of Section \ref{sec:main}, and we  conclude in the last section.
\section{Preliminaries}
\label{sec:prelim}

In this section some background knowledge will be presented in order to understand our work. 

% NOTE: This is 1000% obvious.
%For a detailed explaination the interested reader is suggested to have a look at the cited papers. 

\subsection{Declarative Programming}
The guiding principle of declarative programming is
\begin{center} 
  ALGORITHM = LOGIC + CONTROL.
\end{center}
The idea is that the programmer should only focus on representing the problem without taking care of how to compute the solution of it. Indeed, the latter should be the job of the solver. As an example, a simple logic program \(P_1\) that  is able to append two list and to compute correctly the reversion of a list can be: 
\begin{align}
P_1 \colon \quad%
&append ([], X, X). \\
&append ([X|Y], Z, [X|T ]) \leftarrow append (Y, Z, T ). \\
&reverse([ ], [ ]).\\
&reverse([X|Y ], Z) \leftarrow append (U, [X], Z), reverse(Y, U ). \label{eq:1} \\
\intertext{
where \([\cdot|\cdot]\) is a list constructor.
If we wrote the following rule:}
&reverse([X|Y ], Z) \leftarrow reverse(Y, U ), append(U, [X], Z). \label{eq:2}
\end{align}

instead of the rule~\eqref{eq:1} in program \(P_1\), the meaning of the rule would not change and as a consequence, both programs should have the same behaviour. 

It is desirable that
\begin{enumerate}
\item the order of program rules does not matter
\item the order of subgoals in a rule body does not matter.
\end{enumerate}

\subsection{Stable Model Semantics}
% !!!!! CITATION MISSING BIG TIME !!!!!!
Research on the declarative semantics of negation in logic programming was motivated by the fact that the behavior of SLDNF-resolution \cite{}, adopted by Prolog,  does not fully match the models of programs like \(P_2\):
\begin{align}
  P_2 \colon \quad
&\mathit{pig}((88,34)). \\
&\mathit{easy\_target}(X) \leftarrow pig(X), \text{ not}\: \mathit{hard\_target}(X). \label{eq:3}\\ 
&\mathit{hard\_target}(X) \leftarrow pig(X), \text{ not}\: \mathit{easy\_target}(X). \label{eq:4}
\end{align}
Indeed, SLDNF-resolution will not always terminate when run with this program: For example, given the initial query \(\mathit{easy\_target}(pig((88,34)))\), it is necessary to prove \(\mathit{hard\_target}(pig((88,34)))\) and to so, it is again necessary to prove \(\mathit{easy\_target}(pig((88,34))\) which leads to an obvious cycle.
The non-termination for the Prolog program \(P_2\) with this query does not mean that there is no solution for it. The intuitive models for \(P_2\) are \(\{pig((88,34)), \mathit{hard\_target}((88,34))\}\) or \(\{pig((88,34)), \mathit{easy\_target}((88,34))\}\).

% !!!! WE NEED A CITATION FOR THE STABLE MODEL SEMANTICS HERE !!!! Probably Gelfond and Lifschitz?
\emph{Stable models} of a logic program \(P\) are models of \(P\) which enjoy additional properties according to natural intuitions.
Indeed stable models help us to find the intuitive model of the modified program \(P_1\) and \(P_2\). The idea is guessing an interpretation of the program, and testing its satisfiability. In order to do so, the program from which the model is wanted to be found is first reduced. 

% !!!! WE NEED A CITATION FOR THE REDUCTION !!!!!!
The algorithm for program reduction is:

Given a program \(P\) and an interpretation \(M\), its \emph{reduct} \(P^M\) is a program obtained by 
\begin{enumerate} 
\item removing rules with \(\text{not } a\) in the body for each \(a \in M\)
\item removing literals \(\text{not } a\) from all other rules % for each \(a \notin M\)
\end{enumerate}
After the reduction the least model \(LM(P^M)\) of \(P^M\) is computed and it is tested if \(M = lm(P^M)\). If the equality holds M is called a stable model of \(P\).

If we consider again the program \(P_2\), we could have the following reasonable interpretations.
\begin{align*}
M_1&= \{pig((88,34)), \mathit{easy\_target}((88,34))\}  \\
M_2&= \{pig((88,34)), \mathit{hard\_target}((88,34))\} \\
M_3&= \{pig((88,34)), \mathit{easy\_target}((88,34)), \mathit{hard\_target}((88,34))\} \\
M_4&= \{pig((88,34))\}
\end{align*}

by the definitions stated above it is easy to check that only \(M_1\) and \(M_2\) are stable models. For clarity we show that \(M_1\) is a stable model. First we reduce the program \(P_2\) to \(P_2^{M_1}\) we get the program:
\begin{align}
P_2^{M_1} \colon \quad
&pig((88,34)). \\
&\mathit{easy\_target}(X) \leftarrow pig(X). 
\end{align}
Here, rule~\eqref{eq:4} is removed since \(\mathit{easy\_target}(X) \in M_1\) and \(not\; \mathit{hard\_target}(X)\) is removed from rule~\eqref{eq:3} because \(\mathit{hard\_target}(X) \notin\; M_1\).
\subsection{Answer Set Programming}

\emph{Answer Set Programming} (ASP)~\cite{aspPrime} is a form of declarative programming 
capable of overcoming the limitations of Prolog in the programs \(P_1\) and \(P_2\).
Its high-level approach is depicted in Figure \ref{fig:ASP1} and can be summarized as follows:
\begin{enumerate}
\item Express the problem (instance) as a logic program, i.e.~in such a way that models of the logic program correspond to solutions for the problem.
\item Use an ASP system in order to compute the models of the program.
\item Extract a solution for the problem from the models of the program.
\end{enumerate}
\begin{figure}
  \caption{High-level approach of using Answer Set Programming for declarative problem solving.}
  \begin{center}
    \smartdiagramset{back arrow disabled=true,%
    uniform color list=white for 4 items,%
    uniform arrow color=true,%
    arrow color=gray,%
    arrow line width=0.05cm,%
    text width=1.8cm%
    }
    \smartdiagram[flow diagram:horizontal]%
    {Problem instance \(I\),%
    Encoding: Program \(P\),%
    ASP Solver,%
    Models \& Solutions}    
  \end{center}
  \caption{logical steps made in order to find a stable model of a program.}
  \label{fig:ASP1}
\end{figure}
 
Traditional answer set systems work in two phases:
\begin{enumerate}
\item Grounding: Given a program \(P\) with variables, a (subset) \(P'\) of its grounding is generated which has the same answer sets as \(P\).
% NOTE: Sorry, but this is just wrong on so many levels. The program actually gets exponentially bigger, and the complexity of solving it overall stays the same.
% Grounding is needed because it makes the program smaller and easier to evaluate.
\item Solving: The answer sets of the grounded (propositional) program \(P'\) are computed. First a candidate model is generated and then stability condition is checked.
\end{enumerate}

% !!!!!!!!!!!!! CITATON MISSING !!!!!!!!!!!!!!!

There are different techniques for each of the two steps described and their presentation is out of the scope of this paper. For a more detailed explanation, refer to~\cite{}.

\subsection{HEX-Programs}
HEX-programs~\cite{hex}, \emph{\underline{h}igher-order} logic programs with \emph{\underline{ex}ternal} atoms were introduced as a generalization of extended logic programs under the stable model semantics. The syntax of HEX-programs extend ordinary ASP programs by \emph{external atoms}, which enable a bidirectional interaction between a program and external sources of computation. External atoms have a list of input parameters and a list of output parameters. Informally, to evaluate an external atom, the reasoner passes the instantiated input parameters to the external source associated with the external atom. The external source computes output tuples that are matched with the output list. Formally, an external atom is of the form \(\&g[\vec{Y}](\vec{X})\), where \(\vec{Y} = Y_1, \dotso , Y_k\) are input parameters (from the set of terms, variables and predicates) and \(\vec{X} = X_1, \dotso , X_l \) are output terms.

Note that since predicate symbols may occur as input, this yields a higher order formalism. Implementations of external sources may consider the partial interpretation for input predicates at the time of evaluation.

A way to translate HEX-programs into (usual) first order programs by means of a polynomial reduction \(\Lambda(P)\) is given in \cite{hex}: (1)~Higher order atom of the form \(Y_0(Y_1, \dotso, Y_n)\) are replaced by an ordinary atom \(a_n(Y_0, Y_1, \dotso, Y_n)\). A one-to-one correspondence between stable models of \(\Lambda(P)\) and \(P\) can be shown. (2)~External atoms require a more intricate treatment. We replace external atoms of the form \(\&g[\vec{X}](\vec{Y})\) by \(p_{\&g}(\vec{X},\vec{Y})\), but in the presence of negation as failure or non-monotone external atoms, more machinery is needed which we omit for brevity. For further details on the evaluation of HEX-programs refer to \cite{effeval1,effeval2}.
\section{The \ah Agent}
\label{sec:agent}
% Describe the Angry Birds scenario/setting and how the agent works
\ah is an artificial player for \ab,
which carries out reasoning with respect to
the world knowledge is carried out by
means of computing the answer set of a
HEX-program.
It originated as a re-implementation of
the na\"{i}ve agent provided by the organizers
of the \abc \cite{angryAI},
with general improvements on main components
of it and a complete rewriting of the
\emph{planning} component as a collection
of HEX-programs.

In this section, an overview of the \ah
agent will be presented, with a focus on
the declarative part and its features. 
For a thorough and detailed explanation of
the implementing process, of the agent
layout and additional considerations on
performance, the reader may refer to~\cite{angryhex}.

\subsection{Architecture of the Agent and its Connection to the Game}
\label{sec:agent_base}

The system that executes \ah and \ab consists of a
number of loosely coupled components,
each dealing with different aspects of the interaction.
Since the game itself runs inside a browser, a
\emph{browser extension} that takes images of the game
and forwards input to it, is connected to a \emph{game server}
which mediates between the agent and the game. The game server
is in turn connected to \ah.
The competition organizers prepared some Java code to
simplify bootstrapping an agent implementation, and \ah
makes use of these components, also referred to as \emph{framework}.

Aside from all this auxiliary machinery we are interested in
the core components of the agent that are closely connected to
or implement its reasoning process:
\begin{description}
    \item[Vision] segments the images
    captured from the game environment, returning
    the minimum bounding rectangles of essential
    objects, as well as relevant information,
    like the types of birds available, the material
    of which bricks are composed and where pigs are placed.
    It is part of the framework.
    \item[Trajectory] estimates
    the flight path of birds as they are
    catapulted off the slingshot; here, orientation and distance of
    the release point are taken into account.
    \item[Planning] also called \emph{AI agent} in the
    % What the hell is "the original setting"? Maybe write it in quotes since its just a reference but we are not actually using that term.
    original setting, this part delivers the order
    of the played levels, the choice of birds to use,
    and other strategy-related choices.
\end{description}

\subsection{HEX-programs for Planning the Agent's Actions}

The main contribution of \ah is the \textbf{Planning} component, which has
been rewritten as a collection of HEX-programs.
These are partitioned in two layers, namely
the \textbf{Tactic} and the \textbf{Strategy} layer.
Then, these programs are fed to the
\textsc{dlv-hex} solver~\cite{dlvHEX},
which returns answer sets containing
information about the next moves in the game.

The purpose of the \textbf{Tactic} layer is to compute optimal shots, retrieving information from the current scene in the game and using knowledge modeled by means of a HEX-program \(\mathcal{P}_{AI}\).
This program represents the knowledge of the agent on shootable targets, the estimates of the damage occurring when a target is hit and the priority of the targets, according to the estimated damages.

The \emph{input} of the \textbf{Tactic} layer consists of the program \(\mathcal{P}_{AI}\), together with a collection \(\mathcal{S}\) of logic assertions about the scene, provided by the \textbf{Vision} component.
The \emph{output} of the layer corresponds to the answer sets of \(\mathcal{P}_{AI} \cup \mathcal{S}\), which contain information about which target to hit and what shot is required.

\emph{External atoms} are used in \(\mathcal{P}_{AI}\) mainly to outsource physics-based simulations to a dedicated, external program and to obtain insights about possible shots, consequences of the shots and the geometry of the scene.

The task of the \textbf{Strategy} layer, on the other hand, is to schedule the sequence of played levels. This layer proceeds according to the following scheme:
\begin{enumerate}
    \item First, all available levels are played once;
    \item Then, each level in which the obtained scores has the greatest difference from the current best score is selected --- up to a set threshold of possible attempts;
    \item Next, those levels where \ah scored better than the current best score, with the least difference in score, are selected (up to a certain number of attempts);
    \item To break ties, a random level is selected.
\end{enumerate}

The \textbf{Strategy} layer keeps a track of the scores that were previously achieved and of the objects that were previously chosen as initial targets.
Objects that were previously targeted are excluded; this ensures that the agent is going to employ a new strategy, with each new attempt on the same level.
\section{Complex Predicates~vs.~Complex Objects}
\label{sec:main}

In this section we present how a specific class of occurences of higher order predicates can be translated to first order by introducing more complex terms.

For example, consider the following rule taken from the tactics of \ah:
% https://github.com/DeMaCS-UNICAL/Angry-HEX/blob/4ea72273518dfde6240ece9dadea927fd1b79c3c/dlv/tactic.dlv#L133
$$ canPush(o_a,o_b) \leftarrow \&canPush[objects,hills](o_a,o_b). \label{main:rule-1} $$

Intuitively, $canPush(o_a, o_b)$ should be true if the object $o_a$ can push the object $o_b$.
To establish the truth of $canPush(o_a, o_b)$ spatial information, like minimum distance and geometric range of the objects, is required; this is obtained by outsourcing the computation to an external source, by means of the external predicate $\&canPush$.
The required knowledge is given as input to $\&canPush$ by using the predicates $objects$ and $hills$: indeed, $\&canPush$ is a \emph{higher-order predicate}.

A useful observation is that the predicates $objects$ and $hills$ both represent sets: $objects$ characterizes the collection of objects that are present in the current level, while $hills$ encodes the morphological structure of the ground on the current level.
These collections represent a form of knowledge that the agent gains before the reasoning phase.
Indeed, in \ah the objects and the information about the world scene are first acquired by the \textbf{Vision} component, as explained in Section~\ref{sec:agent_base}; only after this phase, the \textbf{Planning} component is invoked and the reasoning process takes place.
Thus, for the evaluation of the answer sets w.r.t.\ $canPush$, the elements for which $objects$ and $hills$ do not vary: they are encoded as facts, in the set of assertions \(\mathcal{S}\) added to the program \(\mathcal{P}_{AI}\).

Once understood that $objects$ and $hills$ are representations of sets, we may pose the following question: Would it be possible to replace these predicates with sets?
Or phrased differently: s there a way to have two sets $O$ and $H$, respectively the sets of objects and hills, in place of  $objects$ and $hills$ in the call to $\&canPush$?

The ASP system underlying \ah does not provide support for complex data structures as objects --- thus, sets as objects are not allowed ---, but only symbolic constants, strings and natural numbers.
By lifting such a restriction, we may be able to translate the rule~\eqref{main:rule-1}, containing a higher-order predicate, into one where only first-order predicates appear, such as:
$$ canPush(o_a,o_b) \leftarrow objects(O), \ hills(H), \ \&canPush[O,H](o_a,o_b). \label{main:rule-2} $$

Now, the predicates $objects$ and $hills$ are unary predicates with a different role: their interpretation will only contain one complex object, a set.
Applying this kind of translation to all the rules employing higher-order predicates yields a program where only first-order predicates appear.

\subsection{Translation}

\ah has been initially implemented to work with the \textsc{dlv-hex} solver~\cite{dlvHEX}, which successfully handles higher-order predicates.
Our ASP system of choice, \al (introduced in~\cite{blend}), despite lacking this feature, extends the range of supported data structures beyond those included in \textsc{dlv-hex} --- symbolic constants, natural numbers and strings --- by allowing for terms to be interpreted as Java objects.
The proposed translation exploits this feature: if one can convert the information conveyed by the inputs of the external, higher-order predicates into Java objects, then the higher-order predicates are not necessary anymore.

The first phase of the translation consists of the replacements of input predicates into Java objects, as mentioned in the previous paragraph.
In our ongoing example, the two predicates $objects$ and $hills$, representing sets, are replaced by a java class called \texttt{ASPWorld}.
Each instance of this class represents a snapshot of the world, as retrieved by the agent from the current level.
The class contains two \texttt{Collection} variables: one is expected to contain Angry Birds objects (of type \texttt{ABObject}), while the other includes polygonal shapes, in place of $hills$.

Once that the interpretation of the input of the external predicate has been encoded as a Java class, the interpretation of the external predicate has to be defined.
In \al, this is achieved by means of a Java method.
For $\&canPush$, the interpretation is  is specified as a \texttt{@Predicate} instance with type signature
\begin{center}
    \texttt{public Set<List<ConstantTerm>\-> canPush(ASPWorld aspWorld)}
\end{center}
which indicates that the predicate returns a list of constant terms (of the form \((A,B)\)), that represent the fact that \(A\) can push \(B\), after being hit.

Finally, it is possible to rewrite the ASP program, eliminating all the sources of higher-order predicates.
In our setting, the clause containing the higher-order predicate in~\eqref{main:rule-1} is replaced by the clause

$$ canPush(A,B) \leftarrow world(W), \&canPush[W](A,B) \label{main:rule-3} $$

Here, $\&canPush$ is no longer a higher-order predicate: indeed, it receives a ground atom $W$ as input --- which is interpreted as a \texttt{ASPWorld} instance --- and returns a list of tuples of constant terms $(A,B)$ in which the item $A$ can effectively push the item $B$.

\subsection{About the translation}

While one may be tempted to label the described procedure as a general technique to achieve second-order elimination, it may be useful to investigate what made the proposed conversion possible and to discuss whether it is always the case for such a translation to be applicable.

In the setting of \ah, the use of higher-order predicates was motivated by the need to collect multiple items of the same kind (birds, bricks, pigs, \ldots) as input of the external predicates.
Since input predicates were used as collectors --- hence, used as auxiliary data structures --- the availability of terms' interpretation as Java classes, provided by \al, made it possible to resort to first-order predicates with terms interpreted as suitable Java classes.

However, this approach may not be applicable to other settings, where higher-order predicates are used in a different way.
Thus, we do not claim this to be a general technique for second-order elimination.

% If it makes the cut for next time, we may briefly talk about complexity.
\section{Conclusion}
\label{sec:conc}

\bibliography{ref}
\end{document}
